%%
%% forked from https://gits-15.sys.kth.se/giampi/kthlatex kthlatex-0.2rc4 on 2020-02-13
%% expanded upon by Gerald Q. Maguire Jr.
%% This template has been adapted by Anders Sjögren to the University
%% Engineering Program in Computer Science at KTH ICT. Adaptation is the
%% translation of English headings into Swedish as the addition of Swedish
%% text. Original body text is deliberately left in English.


%% set the default language to english or swedish by passing an option to the documentclass - this handles the inside tile page
%\documentclass[english]{kththesis}


% \usepackage[style=numeric,sorting=none,backend=biber]{biblatex}

\setlength {\marginparwidth }{2cm} %leave some extra space for todo notes
\usepackage{todonotes}

\usepackage[perpage,para,symbol]{footmisc} %% use symbols to ``number'' footnotes and reset which symbol is used first on each page

%% Reduce hyphenation as much as possible
\hyphenpenalty=15000
\tolerance=1000

%%----------------------------------------------------------------------------
%%   pcap2tex stuff
%%----------------------------------------------------------------------------
%\usepackage[dvipsnames*,svgnames]{xcolor} %% For extended colors
\usepackage{tikz}
\usetikzlibrary{arrows,decorations.pathmorphing,backgrounds,fit,positioning,calc,shapes}
\usepackage{pgfmath}	% --math engine


\usepackage{tabularx} % Table
\usepackage{booktabs} % TopRule usage
\usepackage{multirow}
\usepackage{array}


%% some additional useful packages
\usepackage{rotating}		%% For text rotating
\usepackage{array}		%% For table wrapping
\usepackage{graphicx}	        %% Support for images
\usepackage{float}		%% Suppor for more flexible floating box positioning
\usepackage{mdwlist}            %% various list-related commands
\usepackage{setspace}           %% For fine-grained control over line spacing
\usepackage{listings}		%% For source code listing
\usepackage{bytefield}          %% For packet drawings
\usepackage{tabularx}		%% For simple table stretching
\usepackage{multirow}	        %% Support for multirow colums in tables

\usepackage{url}                %% Support for breaking URLs
\usepackage{hyperref}
\usepackage[all]{hypcap}	%% prevents an issue related to hyperref and caption linking
%% setup hyperref to use the darkblue color on links
\hypersetup{colorlinks,breaklinks,
            linkcolor=darkblue,urlcolor=darkblue,
            anchorcolor=darkblue,citecolor=darkblue}


%% If you are going to include source code (or code snippets)
\usepackage{listings}
%%\usepackage[cache=false]{minted} %% For source code highlighting
%%\usemintedstyle{borland}

\usepackage{csquotes} % Recommended by biblatex






\usepackage{setspace} % space for revision
%\singlespace  % interlinea singola
%\onehalfspace  % interlinea 1.5
%\doublespace  % interlinea doppia

\usepackage{indentfirst} 	% Always indent the paragraph
\usepackage{graphicx}		% Support for images
\usepackage{float}			% Suppor for more flexible floating box positioning
\usepackage{setspace}   	% For fine-grained control over line spacing
\usepackage{textcomp}   	% Angle symbol
\usepackage{url}       		% Support for breaking URLs
\usepackage{hyperref}		% Link things in the document
\usepackage[all]{hypcap}	% Prevents an issue related to hyperref and caption linking
\hypersetup{colorlinks,     % Cliccable things are darkblue -> color need to be defined
			breaklinks,
			linkcolor=darkblue,
			urlcolor=darkblue,
			anchorcolor=darkblue,
			citecolor=darkblue}
\usepackage[perpage,para,symbol]{footmisc} % use symbols to ``number'' footnotes and reset which symbol is used first on each page
%%%%%%%%%%%%%%%%%%%%%%%%%%%%%%%%%%%%%%%
%%%%%%%%%%%%%%%%%%%%%%%%%%%%%%%%%%%%%%%
% Tikz package
\usepackage{tikz,fp,ifthen}
\usepackage{pgfmath}
\usetikzlibrary{backgrounds}
\usetikzlibrary{decorations.pathmorphing,backgrounds,fit,calc,through}
\usetikzlibrary{arrows}
\usetikzlibrary{decorations,shadows}
\usetikzlibrary{fadings}
\usetikzlibrary{patterns}
\usetikzlibrary{mindmap}
\usetikzlibrary{decorations.text}
\usetikzlibrary{decorations.shapes}

\usetikzlibrary{shapes.geometric, arrows}
\usetikzlibrary{arrows.meta, positioning}
\usetikzlibrary{arrows}

\tikzstyle{startstop} = [rectangle, rounded corners, minimum width=3cm, minimum height=1cm, text width=3cm,text centered, draw=black] %, fill=red!30
\tikzstyle{process} = [rectangle, minimum width=3cm, minimum height=1cm, text width=3cm, text centered, draw=black] % , fill=orange!30
\tikzstyle{decision} = [circle, thick, minimum width=3cm, minimum height=1cm, text width=3cm, text centered, draw=black] %, fill=green!30]

\tikzstyle{arrow} = [thick,->,>=stealth]
\tikzstyle{line} = [thick,>=stealth]
%%%%%%%%%%%%%%%%%%%%%%%%%%%%%%%%%%%

%%%%%%%%%%%%%%%%%%%%%%%%%%%%%%%%%%%
% Captions - fix it, it shouldn't be only centered
\usepackage{caption}
\DeclareCaptionFormat{citation}{%
	\ifx\captioncitation\relax\relax\else
	\captioncitation\par
	\fi
	#1#2#3\par}
\newcommand*\setcaptioncitation[1]{\def\captioncitation{\centering\textit{Source:}~#1}}
\let\captioncitation\relax
\captionsetup{format=plain, font=small, labelfont=bf, justification=justified} % format=citation for working with source
%%%%%%%%%%%%%%%%%%%%%%%%%%%%%%%%%%%
%%%%%%%%%%%%%%%%%%%%%%%%%%%%%%%%%%%
% New line / New paragraph paramethers
\hyphenpenalty=15000 			% No words finiscing and restrating after -
\tolerance=1000
\setlength{\parindent}{24pt}    % Paragraph indentation and paragraph spacing
\setlength{\parskip}{0em}
%%%%%%%%%%%%%%%%%%%%%%%%%%%%%%%%%%%
%%%%%%%%%%%%%%%%%%%%%%%%%%%%%%%%%%%
%% New colors, darkblue is useful in hyperref
\definecolor{darkblue}{rgb}{0.0,0.0,0.3}
\definecolor{darkred}{rgb}{0.4,0.0,0.0}
\definecolor{red}{rgb}{0.7,0.0,0.0}
\definecolor{lightgrey}{rgb}{0.8,0.8,0.8}
\definecolor{grey}{rgb}{0.6,0.6,0.6}
\definecolor{darkgrey}{rgb}{0.4,0.4,0.4}
\definecolor{aqua}{rgb}{0.0, 1.0, 1.0}
%%%%%%%%%%%%%%%%%%%%%%%%%%%%%%%%%%%
%%%%%%%%%%%%%%%%%%%%%%%%%%%%%%%%%%%
%% Acronyms
% note that nonumberlist - removes the cross references to the pages where the acronym appears
% note that nomain - does not produce a main gloassay, this only acronyms will be in the glossary
% note that nopostdot - will present there being a period at the end of each entry
\usepackage[acronym, section=section, sort=def, nonumberlist, nomain, nopostdot]{glossaries}
\glsdisablehyper
\makeglossaries
%%% Local Variables:
%%% mode: latex
%%% TeX-master: t
%%% End:
% note the use of a non-breaking dash in long text for the following acronym
\newacronym{IQL}{IQL}{Independent Q‑Learning}

\newacronym{LAN}{LAN}{Local Area Network}
% note the use of a non-breaking dash in the following acronym
\newacronym{WiFi}{Wi-Fi}{Wireless Fidelity}

\newacronym{WLAN}{WLAN}{Wireless Local Area Network}
\newacronym{UN}{UN}{United Nations}

\newacronym{SDG}{SDG}{Sustainable Development Goal}

\newacronym{SLAM}{SLAM}{Simultaneous Location and Mapping}
\newacronym{ROS}{ROS}{Robotic Operating System}
\newacronym{WO}{WO}{Wheel Odometry}
\newacronym{VO}{VO}{Visual Odometry}
\newacronym{VIO}{VIO}{Visual-Inertial Odometry}
\newacronym{ICP}{ICP}{Iterative Closest Point}
\newacronym{LRF}{LRF}{Laser Range Finder}
\newacronym{DR}{DR}{Dead Reckoning}
\newacronym{UGV}{UGV}{Unmanned Ground Vehicles}
\newacronym{MEMS}{MEMS}{MicroElectroMechanical System}

\newacronym{RMSE}{RMSE}{Root Mean Square Error}
\newacronym{UWB}{UWB}{UltraWideBand}

\newacronym{EPOS}{EPOS}{Exact Positioning Operating System}
\newacronym{DOP}{DOP}{Dilution Of Precision}
\newacronym{HDOP}{HDOP}{Horizontal DOP}

\newacronym{F2F}{F2F}{Frame to Frame}

\newacronym{SIFT}{SIFT}{Scale-Invariant Feature Transform}
\newacronym{SURF}{SURF}{Speeded-Up Robust Features}
\newacronym{ORB}{ORB}{Oriented FAST and Rotated BRIEF}


\newacronym{ICC}{ICC}{Instantaneous Center of Curvature}

\newacronym{NRAO}{NRAO}{National Radio Astronomy Observatory}


\newacronym{NMEA}{NMEA}{National Marine Electronics Association}


\newacronym{API}{API}{Application Programming Interface}



\newacronym{2D}{2D}{2 Dimensional}
\newacronym{3D}{3D}{3 Dimensional}

\newacronym{IMU}{IMU}{Inertial Measurement Unit}
\newacronym{GPS}{GPS}{Global Positioning System}
\newacronym{GPSRTK}{GPS-RTK}{GPS Real Time Kinematic}
\newacronym{KF}{KF}{Kalman Filter}
\newacronym{PF}{PF}{Particle Filter}
\newacronym{EKF}{EKF}{Extented Kalman Filter}
\newacronym{IEKF}{IEKF}{Iterated Extented Kalman Filter}
\newacronym{AEKF}{AEKF}{Adaptive Extented Kalman Filter}
\newacronym{UKF}{UKF}{Unscented Kalman Filter}
\newacronym{UT}{UT}{Unscented Transform}

\newacronym{HRP}{HRP}{Husqvarna Research Platform}
\newacronym{GNSS}{GNSS}{Global Navigation Satellite System}
\newacronym{INS}{INS}{Inertial Navigation System}



\newacronym{ECEF}{ECEF}{Earth-Centered Earth-Fixed}
\newacronym{ECR}{ECR}{Earth Centered Rotation}
\newacronym{WGS}{WGS}{World Geodetic System}
\newacronym{ENU}{ENU}{East-North-Up}

\newacronym{RPi}{RPi}{Raspberry Pi}

\newacronym{RGB}{RGB}{Red Green Blue}
\newacronym{RGBD}{RGB-D}{RGB-Depth}
\newacronym{RTABMAP}{RTAB-Map}{Real-Time Appearance-Based Mapping}

\newacronym{LMSE}{LMSE}{Least Mean Squared Error}


\newacronym{DOF}{DOF}{Degree of Freedom}

\newacronym{ALM}{ALM}{Autonomous Lawn Mower}
 % load the acronyms file
%%%%%%%%%%%%%%%%%%%%%%%%%%%%%%%%%%%
%%%%%%%%%%%%%%%%%%%%%%%%%%%%%%%%%%%
%% definition of new command for bytefield package
\newcommand{\colorbitbox}[3]{%
	\rlap{\bitbox{#2}{\color{#1}\rule{\width}{\height}}}%
	\bitbox{#2}{#3}}

%\usepackage[pdfpagelabels]{hyperref}


% Command to write comments
\newcommand{\com}[1]{}


%\usepackage[ruled,vlined]{algorithm2e}
%\usepackage{algorithm2e}
\usepackage{algorithm}
\usepackage{algorithmic}

%%% Coloring the comment as blue
%\newcommand\mycommfont[1]{\footnotesize\ttfamily\textcolor{blue}{#1}}
%\SetCommentSty{mycommfont}

%\SetKwInput{KwInput}{Input}                % Set the Input
%\SetKwInput{KwOutput}{Output}              % set the Output


\usetikzlibrary{arrows.meta,
	chains,
	positioning,
	shapes.geometric,
	shapes,positioning
}


\usepackage{subcaption}

\usepackage{svg}

% Uncertainty on the table
\sisetup{separate-uncertainty}




\DeclareUnicodeCharacter{2212}{-}
