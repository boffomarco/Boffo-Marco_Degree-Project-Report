\chapter{Discussions}
\label{ch:discussion}

\com{
\todo[inline]{This can be a separate chapter or a section in the previous chapter.}
The results highlights that it is possible to achieve the goals defined in chapter %\ref{ch:introduction}.
}


\noindent After this master thesis, two related topics have been investigated.
The findings are discussed below.


\section{Localisation}
\noindent
Using a sensor fusion approach, such as \gls{AEKF}, it is possible to reach fast and robust localisation estimates.

Not all the sensors behaved as expected and the best configuration has not been achieved using all the sensors available as expected.

\section{Mapping}
\noindent
With the improved localisation, it is possible to limit the mowing area of the \gls{ALM} using a virtual boundary.

Using collision sensors and accounting for the robot pose uncertainty it is possible to provide a better understanding of the robot environment.


\section{Limitations}
\label{sec:limitations}
\com{
\todo[inline]{What did you find that limited your
  efforts? What are the limitations of your results?}
}
\noindent As the system has been tested experimentally on the field, the conditions to run it outside were needed.
This limited the available amount of time to do outdoor testing in a relevant matter, especially given the raining days.

Being the system an hardware implementation, different aspects needed to be tested before running the tests, removing time to more advanced tuning of the implementation.


\cleardoublepage
%\clearpage

