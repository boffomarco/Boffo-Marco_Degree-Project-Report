
% Make first version of the introduction section consist of four
% paragraphs. Add paragraphs later if needed.

\chapter{Introduction}
\label{ch:introduction}

% usually 2-4 pages, you can use ``I'' or "We'' when you
% describe what you have done


% 1. Introduction to the problem and problem statement:
% why is the work needed/done, how will ``the world'' benefit from it.
% everybody with some education should be able to read this part
% and understand that the thesis is useful
% you can use sections for these parts\section{Motivation and Problem Statement}

% 2. how do you attack the problem including the method (analysis,
% experimentation etc)

% 3. other possible approaches or solutions

% 4. describe your solution and the summarize the main results,
%    if possible the research contributions

% 5: Thesis structure: outline of rest of thesis

\com{

    This\todo{The first paragraph after a heading is not indented, all of the
      subsequent paragraphs have their first line indented.} chapter describes the
    specific problem that this thesis addresses, the context of the problem, the
    goals of this thesis project, and outlines the structure of the thesis.\\

    \todo[inline]{Give a general introduction to the area. (Remember to use appropriate references in this and all other sections.)}

    at \SI{20}{cm}.
}

\noindent
Autonomous systems development has improved consistently in the later period, providing improvements to every day life.
These systems are able to autonomously execute their tasks without human interaction and to adapt to dynamic settings.
On the field of robotics, their support is relevant when tasks are repetitive and the human presence is not necessary.
In the specific case of mobile robotics, autonomous navigation is required and those systems can achieve it using knowledge of their pose and surroundings.

The pose of a mobile robot can be estimated using sensors which provide measurements of their environment.
Multiple types of sensors are available, which differ in frequency, phenomenon captured, and reliability.
Heterogeneous sensors provide measurements which need to be properly fused together to improve the individual knowledge that they provide.

The awareness and the precision related to the position in the working environment changes with respect to each application.
The first autonomous mobile robots available to the consumers were the indoor cleaning robots, where simple collision detection is enough to wander in the indoor environment.
A more recent consumer application of such robots is the \gls{ALM}, which works in the more complicated outdoor setting.

\glspl{ALM} mow the lawn using a random walk coverage approach~\cite{karol_ardic_conditional_2016}, i.e. they move in a random direction until they detect a specific underground wire or until they collide with an object.
They will then rotate towards another direction and keep on going with this behavior.
To do so, they rely in a minimal set of sensors to achieve their goals in dynamic outdoor environments.

Boundary wires are laid underground to provide the outer limits of the lawn working area, and some guide wires are used to aid the mobile robot with more complex tasks that the random behaviour will not be able to address, such as the autonomous return to the charging base and the navigation through narrow passages.
These wires transmit a proprietary electrical current signal passing through and the mobile robot is able to detect it using three different magnetometer sensors, one positioned in the center front and the other two placed on each side.
Through the analysis of the magnetic field direction and pulses, the lawn mower is able to understand if the wire defines a boundary or if it provides a guide to go towards another area of the lawn or back to the charging base.
However, these external wires require regular maintenance which could be avoided with a more sophisticated model.

Additionally, the \glspl{ALM} implement collision sensors, located near the rear wheels.
When the mobile robot bumps into any firm object from the front, it will trigger push sensors situated on the chassis.
It has been used to react in case of collision with unexpected objects, but eventually it could be used to map the environment with those objects and avoid them systematically.

In more sophisticated and recent models, additional frontal ultrasonic range sensors are installed to slow down the mobile robot before potentially colliding with objects in its trajectory.

The most advanced models also include \Glspl{GNSS} receivers to keep track of the global robot position, for theft protection and lawn monitoring purposes.
However, it is not precise enough on its own to provide real time localisation information for it to be used for navigation purposed by the \gls{ALM}, but it can be used to detect that a particular side of the lawn has not been mowed in a while.% using the random walk coverage approach.

This configuration does not require them to understand their position on the world, as they just use a reactive architecture defined by a random behaviour motion pattern without any understanding of their surroundings~\cite{wooldridge_agent_1995}.
These robots are situated, i.e. they are not taking into account events of the past and they cannot foresee their future interactions with the environment  ~\cite{muller_1999}.
As such, they are not able to plan ahead their path and they just react to perceptions of their surroundings.
They do not have a model of their world, and they do not need to update it in case of unexpected changes of their environment~\cite{wooldridge_agent_1995}.
They are flexible and adaptive as they rely on the infrastructure manually installed for them~\cite{wahde2012introduction},  which however requires additional installation and maintenance.


\com{
\section{Background}
\label{sec:background}
%Present the background for the area. Set the context for your project – so that your reader can understand both your project and this thesis. (Give detailed background information in Chapter 2 - together with related work.)
%Sometimes it is useful to insert a system diagram here so that the reader knows what are the different elements and their relationship to each other.
%This also introduces the names/terms/… that you are going to use throughout your thesis (be consistent). This figure will also help you later delimit what you are going to do and what others have done or will do.
}

\noindent 


\section{Problem}
\com{
Longer problem statement\\
If possible, end this section with a question as a problem statement.
}

\noindent
The current method of autonomously mowing the lawn is not effective, since it is based on an external infrastructure to perform a reactive behaviour based on a random walk coverage algorithm~\cite{karol_ardic_conditional_2016}.
The problems that arise from this setting are multiple, as the external components require maintenance and the current algorithm cannot guarantee a complete deterministic coverage of the area.
%Random planning cannot guarantee a complete coverage, whereas, many deterministic techniques are not solely eligible for unstructured outdoor environments, since they highly suffer from wheel slippage or numerical drift. Besides, complete coverage techniques either demands high computational power or expensive sensor hardware.
This configuration for the \glspl{ALM} was given by constraints related to the low computational power available in the past.
Moreover, the recent progresses in the performance of embedded devices and sensors have made the navigation model of \glspl{ALM} obsolete, and as those components have also lowered their cost, their implementation is now feasible even for this consumer products.
The combined availability of more advanced devices and more computational power enables for the implementation of real time applications to improve the performance of these mobile robots.


The performance of an \gls{ALM} can be improved with a more advanced perception and action system used to implement deterministic techniques of coverage planning.
The problem that needs to be solved is the development of a more precise localisation and mapping module.
The \gls{ALM}, to perceive its surroundings and act accordingly, needs additional computational power and sensors than the ones available in the current configurations.
With a more advanced set of components, it would be possible to reach a more accurate understanding of the mobile robot's pose inside a map of the lawn itself.
An analysis of the best configuration of sensors, along with their measurements fusion, is needed to understand how to provide to costumers a more reliable product which needs no maintenance of the external infrastructure.
The issue is related to finding the most appropriate technique to fuse all these sensors' measures in a setting where every downside is compensated and every useful aspect of those sensors is exploited and highlighted,


An aspect that is worth investigating is related to the removal of the need of external infrastructure used for localisation purposes.
Instead of relying to a vulnerable boundary wire and with a set of sensors built in the \gls{ALM} directly, without installing additional systems on the lawn, the implementation of such a system will allow for the offering of a more complete set of features to the users.
Additionally, the removal of such a boundary wire or external infrastructure enable the \gls{ALM} to cover larger areas of custom configuration.

Finally, the fused combination of given control commands and the whole set of sensors at disposal for this project has yet to be investigated in research, as usually fewer sensors are available.


%\subsection{Scientific and engineering issues}
%The scientific relevance of this project is derived from the fact that outdoor localisation and mapping of dynamic environments are still far from reaching a reliable solution.
%This work will aid in the handling of dynamical outdoor settings.
%Since there are just few available devices able to mow the lawn without installing additional and external devices on the lawn, this work will investigate how it can be done using a set of sensors built in the \gls{ALM} directly and removing the need for external infrastructure. This work will provide valuable insights about this heterogeneous sensor fusion.


Autonomously mowing the lawn will help saving time and avoid human intervention as much as possible.
The steps needed to improve current systems are related to a dynamic management of the boundary of the lawn, eliminating the need for a boundary wire with its related installation and maintenance.

The knowledge about position and orientation of the mobile robot, with respect to a map, enables for the application of more deliberative architectures~\cite{genesereth_logical_1987}.
With this architecture, the pose with respect to a map will allow the \gls{ALM} to make independent decisions and to plan its path to cover the lawn in a shorter amount of time, leaving a better pattern, avoiding unexpected objects, and saving energy and resources.
%The term `deliberative agent' seems to have derived from Genesereth's use of the term `deliberate agent' to mean a specific type of symbolic architecture [Genesereth and Nilsson, 1987].) We define a deliberative agent or agent architecture to be one that contains an explicitly represented, symbolic model of the world, and in which decisions (for example about what actions to perform) are made via logical (or at least pseudo-logical) reasoning, based on pattern matching and symbolic manipulation. ~\cite{genesereth_logical_1987}


\section{Research Goals}

\com{
State the purpose  of your thesis and the purpose of your degree project.

Describe who benefits and how they benefit if you achieve your goals. Include anticipated ethical, sustainability, social issues, etc. related to your project. (Return to these in your reflections in Section~\ref{sec:reflections}.)

State the goal/goals of this degree project.

This has been divided into the following three sub-goals:
\begin{enumerate}
\item Subgoal 1 \todo[inline, backgroundcolor=aqua]{för att tillfredsställa problemägaren – industrin?}
\item Subgoal 2\todo[inline, backgroundcolor=aqua]{för att tillfredsställa ingenjörssamfundet och vetenskapen – akademin) }
\item Subgoal 3\todo[inline, backgroundcolor=aqua]{eventuellt, för att uppfylla kursmålen – du som student}
\end{enumerate}

In addition to presenting the goal(s), you might also state what the deliverables and results of the project are.
}

\noindent
This master thesis investigates how to improve the performance of an \gls{ALM} providing localisation and mapping features using a set of heterogeneous sensor directly mounted on it.
It will focus on the implementation and analysis of a module based on the fusion of the measurements provided by different configurations of those sensors to remove the need of relying on external infrastructure.
%An overview of the best settings for a precise localisation is provided
%Different configurations of sensors and techniques to fuse their measurements are investigated to provide an overview of the best setting to improve the localisation performance.
Finally, their configurations are investigated through multiple experiments, and drawbacks and improvements of each sensor are analysed.

The goal of this thesis is to provide an overview on how to develop a precise localisation module for an \gls{ALM}.
An analysis about different configurations of sensors and about how to fuse their measures using sensor fusion filters is presented.
As end result, the \gls{ALM} is able to operate in a specified environment without the need to intervene with the installation of additional infrastructure, but just with the addition of heterogeneous sensors installed directly on the mobile robot.

The focus of this degree project is the precise localisation inside a predefined given map to ensure that the \gls{ALM} stays inside the virtual boundaries defined.
Moreover, a map of the lawn is developed to allow for updates using collision events against unexpected object to provide a more dynamic view of the lawn.
%It will benefit both the industry of \gls{ALM} with the knowledge derived from this thesis, and the consumers will obtain a product which requires less maintenance.

The research questions to be evaluated are the following:
\begin{itemize}
    \item Providing an initial map and location of the \gls{ALM} in it, will the \gls{ALM} be able to stay within the given boundaries with a specified margin for error?
    \item Will the \gls{ALM} be able to correctly update the map by adding or removing eventual obstacles found by navigating in the map through collision detection?
\end{itemize}

In order to evaluate if the above goals have been reached, for an \gls{ALM} made to cover an area of \SI{5000}{\meter\squared}, the following measurable objectives will be adopted:
\begin{itemize}
    \item Identify if the mobile robot is able to stay within the boundaries defined, ensuring that the error in boundary violation will be inside the dimension of the \gls{ALM} itself: around \SI{50}{\cm}.
    These measures are defined to show that the lawn mower will not exceed its given boundaries, risking to damage itself, the lawn, or others.
    \item Identify if the \gls{ALM} is able to return to the parking dock to recharge itself with an offset of a maximum of \SI{1}{\m} after a run of at least \SI{500}{\m}. In this way, it will be relevant to show how the mobile robot is not drifting.
    Moreover, with this results it will be possible to correct eventual offsets once at the starting position.
    \item Identify if the \gls{ALM} is able to localise a collision event within an error of at most $1\%$ of the total distance already travelled, e.g. identify a tree with a maximum offset of \SI{20}{\cm} error after a \SI{20}{\m} run.
    This will be addressed to ensure that the mower is able to update precisely the map with the presence of unexpected objects.
\end{itemize}


\com{
(Return to these in your reflections in Section~\ref{sec:reflections}.)
}

\section{Research Methodology}
\com{
Introduce your choice of methodology/methodologies and method/methods – and the reason why you chose them. Contrast them with and explain why you did not choose other methodologies or methods. (The details of the actual methodology and method you have chosen will be given in Chapter~\ref{ch:methods}. Note that in Chapter~\ref{ch:methods}, the focus could be research strategies, data collection, data analysis, and quality assurance.)\\
In this section you should present your philosophical assumption(s), research method(s), and research approach(es).
}

\noindent
The research methodology~\cite{RESEARCHMETHOD} will follow a quantitative approach: acquiring measurements and using them to validate or not the formulated research questions through quantitative analysis.
The assumptions will follow an objective and realistic paradigm where the final results will be evinced quantifying measures of the observations and gaining a better knowledge of the environment.
The research method adopted will be of experimental nature to understand the cause and effect of the obtained measurements, improving where possible, and of descriptive nature to highlight the characteristics of the obtained system.
A deductive approach will be used to test the theories and draw conclusions about the hypotheses described in the research questions.
The research strategy adopted will be based on data collection through multiple case studies of experimental nature and the collected data will be analysed with computational mathematics.
Statistical analysis will be used to test the quality of the obtained results.
In particular, in this degree project, I will apply and discuss the validity, reliability, replicability, and ethics of those results.

The following tasks are required to achieve the above mentioned objectives.
Literature study is the first step.
The next step is related to the improvement of the available localisation module performance, starting from the refinement of the sensor's drivers.
The most relevant sensor fusion technique identified is implemented to merge heterogeneous sensors' measurements.%, improving the accuracy of the localisation performance.
Afterwards, different configurations of sensors are tested and a phase of tuning their attributes to reach more reliable and valid results is done.
The theoretical validation of the sensor fusion system is performed with the usage of simulations of the sensor's measurements according to the required assumptions.
Some outdoor experiments are performed and checked against the ground truth with some metrics to evaluate the improvements on the localisation performance.
Finally, relying on the localisation results, the definition of initial virtual boundary is going to be provided with a first run of the \gls{ALM}.
The knowledge of the elements inside the obtained map is then improved using collisions events, always relying on the localisation improvements.

\begin{figure}[!ht]
	\begin{center}
		\begin{tikzpicture}[font=\small,thick]

			\node[draw,text centered,fill=green!30,rounded corners,
			align=center,
			minimum width=2.5cm,
			minimum height=1cm,
			] (block0) { System \\ Configuration};

			\node[draw,text centered,fill=cyan!30,rounded corners,
			right=of block0, align=center,
			minimum width=2.5cm,
			minimum height=1cm,
			] (block1) { Localisation \\ Improvements };

			\node[draw,text centered,fill=cyan!30,rounded corners,
			right=of block1, align=center,
			minimum width=2.5cm,
			minimum height=1cm,
			] (block2) { Mapping \\ Features };

			\node[draw,text centered,fill=red!30,rounded corners,
			right=of block2, align=center,
			minimum width=2.5cm,
			minimum height=1cm,
			] (block3) { Path \\ Planning};

			% Arrows
			\draw[-latex] (block0) edge (block1);
			\draw[-latex] (block1) edge (block2);
			\draw[-latex] (block2) edge (block3);
		\end{tikzpicture}
	\caption[Caption]{Main tasks of the project:~\\
	Improvement[green], new development[blue], and future development[red].\centering}
	\end{center}
	\label{fig:Maintasks}
\end{figure}


\com{
\subsection{Scope}
Describe the boundary/limits of your thesis project and what you are explicitly not going to do. This will help you bound your efforts – as you have clearly defined what is out of the scope of this thesis project. Explain the delimitations. These are all the things that could affect the study if they were examined and included in the degree project.
\noindent
}

Since such \glspl{ALM} are consumer products, some constraints will be taken into account, such as the configuration complexity and the sensors' cost analysis.
Some aspects regarding this degree project are described here:
\begin{itemize}
    \item The operation area is projected into a \Gls{2D} environment.
    \item The usage of an embedded device, such as a \gls{RPi}, limits the computational power of the module. Thus, the performance might not be as good as if a more powerful device would be used. %In this project, this aspect of optimization of the limited resources available on an embedded system will not be investigate.
    \item A complete \gls{SLAM} algorithm is too computationally expensive to perform well in real-time with considering a dynamic outdoor environment.
    Instead, an easier approach is to split the two different phases.
    The localisation aspect will be achieved with a sensor fusion approach and the mapping happens after localisation, using collision events to update the knowledge of the predefined map.%receiving directly a predefined map layout configuration of the environment with a given starting point of the \gls{ALM} should ease the process of  within it.
    \item Relying on a camera in outdoor settings means that the weather conditions and time of mowing can affect the results.
    As such, the testing will be performed in similar configurations, and, if possible, with an attitude towards limiting such a characteristic.
    \item A Range Finder Sensor was not considered as most of the outdoor environment will be sparse or empty.
    Thus, the sensor would not be able to provide valuable improvement to justify the expensive choice.
    Moreover, such a sensor requires a relevant amount of computational power to be run.
\end{itemize}
They provide the rationale for the project to investigate some aspects more than others.
Some of them are directly limiting and they are not be investigated for such a project.
Some others instead are part of the future works, discussed at the end of this thesis.


\section{Structure of the thesis}
\noindent In the following chapters the topics of this thesis are discussed as follows.

%Chapter~\ref{ch:introduction}, Introduction, presents briefly the background, the problem, and how it will be solved.
%The chapter also presents an overview about competitors' approaches regarding improvements of \gls{ALM} performance.

Chapter~\ref{ch:background}, Background, presents relevant information about mobile robots, localisation, and mapping.
Some theoretical related works are presented to describe the state-of-the-art.
It provides a summary about the lesson learned during the literature review.

Chapter~\ref{ch:methods}, Methods, defines the localisation and mapping approaches chosen from the literature study.
Moreover, the methodologies and experiments configurations to achieve the research goals are explained.

Chapter~\ref{ch:whatYouDid}, Implementation, presents the system implementation adopted to test the system.
It also motivates and elaborates on the implementation of the hardware and software required to run the system.

Chapter~\ref{ch:results}, Experiments and Results, defines the experiments developed to address the research questions.
Moreover, the quantitative results obtained from them are shown to highlight the desired aspects.

Chapter~\ref{ch:discussion}, Discussions, explains and interprets the final results.
The findings and their limitations are elaborated on.

Chapter~\ref{ch:conclusion}, Conclusions, wraps up the thesis and  how to improve further are provided.
Finally, some reflections related to this thesis are presented.


\cleardoublepage
%\clearpage
