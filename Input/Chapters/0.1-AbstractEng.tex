
\begin{abstract}
  \markboth{\abstractname}{}

% Make first version of abstract consist of four sentences. Fill in
% more later, if needed.

% First sentence: state the problem.

% Second sentence: describe why this is a problem.

% Third sentence: state your solution.

% Fourth sentence: sketch the consequences of your solution
\com{
\todo[inline]{Keep in mind that most of your potential readers are only going to read your title and abstract. This is why it is important that the abstract give them enough information that they can decide is this document relevant to them or not. Otherwise the likely default choice is to ignore the rest of your document.\\
A abstract should stand on its own, i.e., no citations, cross references to the body of the document, acronyms must be spelled out, …\\
Write this early and revise as necessary. This will help keep you focused on what you are trying to do.}

Write an abstract\todo{Use about 1/2 A4-page (250 and 350 words).}  with the following components:
\begin{itemize}
  \item What is the topic area? (optional) Introduces the subject area for the project.
  \item Short problem statement
  \item Why was this problem worth a Master’s thesis project? (i.e., why is the problem both significant and of a suitable degree of difficulty for a Master’s thesis project? Why has no one else solved it yet?)
  \item How did you solve the problem? What was your method/insight?
  \item Results/Conclusions/Consequences/Impact: What are your key results/conclusions? What will others do based upon your results? What can be done now that you have finished - that could not be done before your thesis project was completed?\todo[inline]{The presentation of the results should be the main part of the abstract.}
\end{itemize}
}
DRAFT

\noindent
Autonomous lawn mowers have been available to consumers for more than $20$ years now.
During this lapse of time, however, their configuration and performance have seen little improvements.

With the advancement in embedded devices and sensors, such a configuration is now obsolete.
The features provided now are limited and still prone to errors.
Moreover, the installed external infrastructure on which they rely, i.e. the boundary wire, requires maintenance.

This thesis analyses the implementation of a localisation and mapping module which allows for the implementation of more features.
Multiple sensors to improve the localisation of the autonomous lawn mower are analysed, and the best configuration is presented.
Additionally, mapping features are implemented over the positioning system to update the understanding of the robot's environment.

This solution could be used to improve the coverage feature of the autonomous lawn mower.
Moreover, the availability of a more precise localisation could be the starting point for more advanced features, e.g. interaction with the lawn.



\subsection*{Keywords}
\noindent
Autonomous Mobile Robot, Sensor Fusion, Localisation, Mapping


\com{
\todo[inline]{Choosing good keywords can help others to locate your paper, thesis, dissertation, … and related work.}

Choose the most specific keyword from those used in your domain, see for example:
ACM's Computing Classification System (2012) or
(2014) IEEE Taxonomy.

Mechanics:
\begin{itemize}
  \item The first letter of a keyword should be set with a capital letter and proper names should be capitalized as usual.
  \item Spell out acronyms and abbreviations.
  \item Avoid "stop words" - as they generally carry little or no information.
  \item List your keywords separated by commas (",").
\end{itemize}
Since you should have both English and Swedish keywords - you might think of ordering them in corresponding order (i.e., so that the nth word in each list correspond) - thus it would be easier to mechanically find matching keywords.
}

\end{abstract}


%\cleardoublepage
\clearpage
