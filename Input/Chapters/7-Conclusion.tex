\chapter{Conclusions}
\label{ch:conclusion}

\com{
\todo[inline]{Describe the conclusions (reflect on the whole introduction given in Chapter 1).}

	Discuss the positive effects and the drawbacks.\\
	Describe the evaluation of the results of the degree project.\\
	Did you meet your goals?\\
	What insights have you gained?\\
	What suggestions can you give to others working in this area?\\
	If you had it to do again, what would you have done differently?\\
}

\noindent
This project focused on the improvement of the localisation module and the development of a mapping feature.
The conclusions drawn from the results are discussed here, along with a description of some aspects limiting the project.
Moreover, some future works that could performed to improve and continue developing this project are presented.
Finally, some considerations related to the impact of such a project's implementation are discussed.

Regarding the localisation topic, an \gls{EKF} was defined to fuse the measurements of multiple and heterogeneous sensors.
This technique allows for the management of their dynamic noises with an adaptive approach.
By using this technique, different configurations have been analysed.
Through their comparison, the better possible model was found to be the one merging the measurements obtained by the wheel encoders, by the \gls{GPS} receiver, and by the \glspl{IMU}.
The main contributions from the sensors can be summarised as follows.
The wheel encoder provides fast and reliable estimates on the velocities, which are then corrected by the slow and absolute measurements of the \gls{GPS} receiver. The \gls{IMU} improves the estimates of the angular velocity and its measurements on the acceleration are used to provide a more accurate and adaptive measure for the noise of the system.
This configuration provides measurements for each state of the pose, hence the system is observable.
The \gls{EKF} results can be defined as an estimate of the state, simply averaging the measurements obtained using the adaptive noise to account for the dynamical behaviour of the \gls{ALM}.
This last feature enable the \gls{EKF} to consider more the measurements when the acceleration is higher, while moving, and to be less affected by them in case of lower acceleration.

The exploitation of both the control and the additional camera didn't contribute to provide a more precise localisation.
The control is limited by the difference between the desired velocities and the actual ones, and related to their different time of actuation and control compensation.
The \gls{RGBD} camera needs more computational power to properly perform, and it is still subject to system outages, i.e., when the sensor is not providing measurements, in case of sparse environment affected by powerful light sources, e.g., sun. 

Using the best estimates obtained through real experiments and linking them to the research questions, it is clear that the results are not as good as desired.
The euclidean distance overall error during a run, which was supposed to verify the ability of the \gls{ALM} to stay within is boundaries, is about three times the desired \SI{0.5}{m}, so the localisation estimates would not be accurate enough to use them to move autonomously.
Moreover, the final offset obtained by those experiments is around the same value of \SI{1.5}{m}, which is more than the desired \SI{1}{m} but this distance is enough for the adopted \gls{ALM} to perceive its charging station.

The mapping feature has been achieved using the better pose estimates, the collision events, and the mapping method as given by a remote computer.
The module's dependability on the pose estimate, which is not accurate enough, limited its own performance to a point that the related research question has not been validated.
However, it was interesting to notice the flexibility of the Occupancy Grids to provide on the features needed by the system.
Relating these obtained maps to the mapping research question analysed during this project, it can be said that the final results satisfy the objective of setting a virtual boundary and of storing obtacles' positions.
The virtual boundaries have been set with initial runs to set the related cells to negative values.
Also, the posterior estimates managed to improve the knowledge of the system about eventual collision events caused by unexpected objects, delivering a less entropic view of the environment.
However, as aforementioned, those obtained Occupancy Grids projections are not entirely faithful representations of the perceived environment.
Finally, it is important to show how this mapping approach provided another method to test and evaluate the estimates obtained by the localisation module.



\section{Limitations}
\com{
\todo[inline]{What did you find that limited your
  efforts? What are the limitations of your results?}
}
\noindent
As the system has been tested experimentally, some conditions were needed to run it safely without risking to damaging its components.
This limited the available amount of time to do outdoor testing in a relevant matter, especially given the raining days.
Moreover, as the system is an hardware implementation which required accurate configuration, different aspects needed to be tested extensively before running the experiments. 
This aspect left fewer time for the more advanced tuning of the implementation, as the set up of multiple tests and the control of their configuration were needed.

A measurement tape and accurate definition of the path have been used to define the ground truth.
As this aspect is vital to be able to properly evaluate the system, this procedure required a considerable amount of time for each test.
A more accurate way to compute the ground truth might be less limiting for executing the evaluation of future experiments in an easier manner.

Finally, the usage of the \texttt{rosbag} utility of \gls{ROS} during the project to re-use the obtained measurements of the experiments to evaluate different configurations could have affected the results.
Storing of all the measurements is computationally expensive at such high frequencies, and the observations of the system needed for evaluation limited its performances.


\section{Future Work}
\label{sec:futureWork}
\com{
\todo[inline]{Describe valid future work that you or someone else could or should do.\\
Consider: What you have left undone? What are the next obvious things to be done? What hints can you give to the next person who is going to follow up on your work?
}
}
\com{
Due to the breadth of the problem, only some of the initial goals have been
met. In these section we will focus on some of the remaining issues that
should be addressed in future work. ...
}

\com{
In particular, the author of this thesis wishes to point out xxxxxx remains as
a problem to be solved. Solving this problem is the next thing that should be
done. ...
}

\noindent 
The system could be improved with multiple enhancements and expansions.
Possible options are available in multiple directions for both the configuration, localisation, mapping, and path planning aspects.

The current \textit{configuration} of the differential drive mobile robot could be improved by a more sophisticated model.
A four wheel drive \gls{ALM} would provide more general stability for the robot and the adoption of four wheel encoders could provide more reliable measurements for the localisation module.

%\subsection{Localisation}

Related to the \textit{localisation}, the low accuracy on the \gls{GNSS} measurements is one of the main limiting factors.
As the current state of the \glspl{GNSS} constellation is limited, little improvements could be done.
It is noteworthy to mention that from year 2024, new \gls{GNSS} services would be available, such as Galileo High Accuracy Service \cite{noauthor_galileo_nodate} and GPS III.
Those services will aim at delivering Precise Point Positioning for commercial purposes.
The horizontal accuracy provided will decrease from the current $\sigma_{pos} < \tiny{\sim} 2m$ to an estimate of $\sigma_{pos} <  \tiny{\sim} 0.5cm$, offering the possibility to avoid the usage of expensive \gls{GPSRTK} solutions.

A possible topic to investigate could be the expansion of the states of the system in order to navigate in a 3D environment instead than a simple 2D environment as in this project.
It will be possible to account for slopes in the lawn and it could be interesting to evaluate its influence in the \gls{GNSS} readings.

A more accurate refinition of the \gls{EKF} could be obtained by implementing outliers' detection.
It manages the case of non reliable sensors, however it requires the covariances noises matrices to be ultimately perfected. 
It can be crucial to detect when the sensors are not behaving correctly, avoiding to fuse non relevant information.

Another improvement in the sensor fusion approach could be given by the adoption of the Field Kalman Filter proposed in \cite{FieldKF}.
It allows for the recursive estimation of both the states, their related noise covariances, and the parameters governing them. 
Using this approach, it is suggested that errors inherited by the models adopted could be accounted for over time.
    
A final aspect related to the localisation module could be the intrinsic and extrinsic calibration.
In this project the calibration has been achieved with some test runs.
A more accurate system could include the observability analysis to certain states characterizing some parameters to be calibrated, e.g., wheel diameters or distance between the wheel, which as of now are simply measured.

%\subsection{Mapping}

Related to the \textit{mapping}, the exploitation of depth information from the camera could provide a faster identification of collision events.
Eventually it could provide a map of the environment based on depth information instead of collision events.
Its depth readings, could be used for object detection to identify and avoid the objects even before collision.

Moreover, it could be relevant to investigate a trade off between a \gls{SLAM} approach and the Mapping after Localisation presented in this project.
Some aspects could be related to the computations needed to reach reliable results and the time period to retrieve them.

%\subsection{Planning}


\textit{Path planning} algorithm as in~\cite{coveragePathplanning} and \cite{machines6040046}, can be used to provide coverage features to the autonomous lawn mowers.
A comprehensive overview about coverage is available in \cite{galceran_survey_2013}. It discusses all the available techniques and their related strengths and weaknesses.
More specific approaches to be investigated can be found in \cite{hameed_coverage_2017} and \cite{cabreira_grid-based_2019}.
Regarding the avoidance of the objects for the unexpected detection of objects, a techniques is define in \cite{daltorio_obstacle-edging_2010}.

%The last aspect which could be investigated for a navigation system is locomotion~\cite{autonomous-yasuda}.


\section{Reflections}
\com{
\todo[inline]{What are the relevant economic, social,
  environmental, and ethical aspects of your work?
  }
}
\label{sec:reflections}

\noindent 
This project is related to some non technical aspects presented here.

The \textit{ethics and security impact} of this thesis are important to address given the usage of a camera.
To solve the privacy issue related, the images needed for the visual odometry module are going to be solely used inside the automower and not shared via the \gls{ROS} network.
Those frames are destroyed after their processing, avoiding breach of personal privacy in case of theft of the memory card.
Moreover, the information of the lawn configuration and the presence of obstacle can be considered sensible data, thus the system need to be self-contained, and every information is stored inside the embedded devices.
% online system that doesn't store the history of the states also helps with the security of the information stored.
The outputs that need to be shared, such as the localisation estimates and the updated map, must be encrypted with an asymmetric keys known just to the owners of the \gls{ALM}.
Finally, the SD card containing the data should be encrypted, to avoid the retrieval of information without a password in case of theft.

Some \textit{sustainability aspects} related to this project are interesting to be discussed.
By keeping the localisation module included in the automower, the need to install additional infrastructure will be removed.
The boundary wires currently adopted are susceptible to deterioration, so this external boundary configuration will produce some wastes that the virtual boundary do not have.
This finding may result in a reduction of wastes.
Furthermore, thanks to a more accurate control of the lawn with a given map, it will be possible to dynamically update it and preserve a part of the lawn to allow bees and flowers to prosper, without intervening on the boundary wire.
The bee preservation aspect is vital, and the possibility to set virtual boundaries for the automower could allow bees to find flowers to sustain themselves and prosper, still in a lawn which is otherwise well mowed.


Finally, as this project focuses on the improvements available for a consumer product, the \textit{economic aspect} is important to be analysed.
A comparison between the available and proposed approach is relevant, as a cheaper solution that provides more features might attract more customers.
Simply adding additional IMUs, as in "\textit{Model 7}" of Section \ref{sec:loc-method}, it is possible to get better localisation, and with a more advanced tuning as defined from the Future Work, it could be possible to remove the need of the boundary wire.
As such, by adding to the automower the cost of these additional sensors, it will be possible to save money from both the additional maintenance required by the external infrastrucure and its need of electricity to emit the magnetic field that the automower needs to detect.
Overall, given the high reliability over time of these sensor, the consumers will be able to save money in the long run.
However, the product-market fit might not be the best, since the company that will sell the automower also provides the boundary wire and removing that need will economically lower their revenues in that aspects. 
This aspect needs to be validated against the willingness of the customers to buy a system that requires less maintenance.
So the trade-off will be between the more services offered and the lower revenues from selling additional infrastructures.
%Analysis of costs of sensors and usual boundary wire, plus related maintenance

\com{
The thesis contributes to the \gls{UN}\enspace\glspl{SDG} numbers ... by ...
}


\cleardoublepage
%\clearpage
