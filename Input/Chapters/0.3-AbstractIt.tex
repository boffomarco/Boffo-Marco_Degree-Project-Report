
\selectlanguage{italian}

\begin{abstract}
    \markboth{\abstractname}{}


DRAFT

\noindent
I tosaerba autonomi sono disponibili per i consumatori da più di $ 20 $ anni.
Durante questo lasso di tempo, tuttavia, la loro configurazione e le prestazioni hanno visto piccoli miglioramenti.

Con il progresso dei dispositivi e dei sensori integrati, una tale configurazione è ora obsoleta.
Le funzionalità fornite ora sono limitate e ancora soggette a errori.
Inoltre, l'infrastruttura esterna installata su cui si basano, ovvero il cavo perimetrale, richiede manutenzione.

Questa tesi analizza l'implementazione di un modulo di localizzazione e mappatura che consente l'implementazione di più funzionalità.
Vengono analizzati più sensori per migliorare la localizzazione del rasaerba autonomo e viene presentata la configurazione migliore.
Inoltre, le funzionalità di mappatura vengono implementate sul sistema di posizionamento per aggiornare la comprensione dell'ambiente del robot.

Questa soluzione potrebbe essere utilizzata per migliorare la funzionalità di copertura del rasaerba autonomo.
Inoltre, la disponibilità di una localizzazione più precisa potrebbe essere il punto di partenza per funzionalità più avanzate, ad es. interazione con il prato.

\subsection*{Parole chiave}
\noindent
Robot mobili autonomi, fusione di sensori, localizzazione, mappatura
\end{abstract}

%\cleardoublepage
\clearpage
