
\selectlanguage{italian}

\begin{abstract}
    \markboth{\abstractname}{}


\noindent
I tosaerba autonomi sono disponibili per i consumatori da oltre $ 20 $ anni.
Durante questo periodo, i progressi nei calcoli dei dispositivi integrati e nelle prestazioni dei sensori hanno portato a miglioramenti nell'affidabilità di questi robot.
Nonostante i recenti miglioramenti, l'opportunità di innovazione di tali sistemi rimane significativa.
% ciao amico questo non ha senso per me, ma rimettilo se vuoi Nonostante questo, poco sforzo è stato speso per la loro innovazione ed evoluzione relativa allo sviluppo di nuove funzionalità.

% Perché questo è un problema?
I robot autonomi hanno ancora funzionalità limitate.
Si affidano a fili elettrici installati sottoterra per delimitare i confini del prato, a cui reagiscono senza ragionamento.
Tale configurazione è ormai considerata obsoleta e sono disponibili soluzioni autonome più efficaci.

% indica la tua soluzione.
Questa tesi si concentra sull'utilizzo delle tecniche attualmente disponibili per progettare i moduli principali necessari per far avanzare le capacità di questi sistemi. % (questa frase non dice molto, è abbastanza generica... cosa stai facendo veramente? cioè progetta un filtro kalman per fare xyz)
Vengono presentate l'analisi e la relativa implementazione delle funzionalità di localizzazione e mappatura per i tosaerba autonomi.
Vengono studiati sensori eterogenei e le loro diverse configurazioni e viene proposto un filtro di Kalman adattivo esteso per fondere le loro misurazioni.
Questa tecnica migliora la stima della posa del rasaerba autonomo, che viene poi sfruttata dal modulo di mappatura.
L'approccio scelto per quest'ultimo, basato sull'inferenza bayesiana, riesce ad aggiornare la conoscenza della mappa basata su interazioni dirette con l'ambiente.

% abbozza le conseguenze della tua soluzione
I risultati finali evidenziano l'importanza di una localizzazione precisa come vero collo di bottiglia per lo sviluppo di nuove funzionalità.
La stima della posa migliorata consente la definizione di un confine virtuale. La definizione non è sufficientemente precisa per mappare correttamente la presenza di oggetti nell'ambiente
Esempi di funzionalità avanzate a partire dalla configurazione proposta sono l'implementazione di algoritmi di copertura deterministici e l'interazione con gli oggetti del prato.

\com{
I tosaerba autonomi sono disponibili per i consumatori da più di $20$ anni.
Durante questo lasso di tempo, tuttavia, la loro configurazione e le prestazioni hanno visto piccoli miglioramenti.

Con il progresso nei calcoli dei dispositivi embedded e nelle prestazioni dei sensori, la loro implementazione può essere considerata ormai obsoleta.
Le funzionalità disponibili sono limitate e ancora soggette a errori.
Inoltre, l'infrastruttura esterna installata su cui si basano, ovvero il cavo perimetrale, richiede manutenzione.

Questa tesi analizza l'implementazione di un modulo di localizzazione e mappatura.
Vengono analizzati più sensori e le loro diverse configurazioni per migliorare la stima della posa del tosaerba autonomo.
Le migliori impostazioni vengono quindi adottate con il modulo di mappatura per aggiornare una mappa con la comprensione dell'ambiente del robot.

Questa soluzione è il punto di partenza per funzionalità più avanzate aspetti avanzati.
La disponibilità di una localizzazione e definizione cartografica più precisa consente l'implementazione di algoritmi di definizione della copertura e di interazione con il prato.
}
\subsection*{Parole chiave}
\noindent
Robot mobili autonomi, fusione di sensori, localizzazione, mappatura
\end{abstract}

%\cleardoublepage
\clearpage
