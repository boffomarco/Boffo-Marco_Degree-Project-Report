
\begin{otherlanguage}{swedish}

\begin{abstract}
    \markboth{\abstractname}{}

\com{
    \todo[inline]{All theses at KTH are required to have an abstract in both English and Swedish.\\
If you are writing your thesis in English, you can leave this until the final version. If you are writing your thesis in Swedish then this should be done first – and you should revise as necessary along the way.\\
If you are writing your thesis in English, then this section can be a summary targeted at a more general reader. However, if you are writing your thesis in Swedish, then the reverse is true – your abstract should be for your target audience, while an English summary can be written targeted at a more general audience.\\
This means that the English abstract and Swedish sammnfattning
or Swedish abstract and English summary need not be literal translations of each other.\\

The abstract in the language used for the thesis should be the first abstract, while the Summary/Sammanfattning in the other language can follow.\\

Exchange students many want to include one or more abstracts in the language(s) used in their home institutions to avoid the neeed to write another thesis when returing to their home institution.
}
}

\noindent 
Autonoma gräsklippare har varit tillgängliga för konsumenter i mer än $20 $ år.
Under denna period har framsteg inom beräkningar av inbyggda enheter och sensorprestanda lett till förbättringar av tillförlitligheten hos dessa robotar.
Trots de senaste förbättringarna är möjligheten till innovation av sådana system fortfarande betydande.
% hej kompis det här är inte meningsfullt för mig, men lägg tillbaka det om du vill Trots detta har lite ansträngning ägnats åt deras innovation och utveckling relaterad till utvecklingen av nya funktioner.

% Varför är detta ett problem?
Autonoma robotar har fortfarande begränsade funktioner.
De förlitar sig på elektriska ledningar installerade under jord för att avgränsa gräsmattans gränser, som de reagerar på utan resonemang.
En sådan konfiguration anses nu vara föråldrad och mer effektiva autonoma lösningar finns tillgängliga.

% ange din lösning.
Den här avhandlingen fokuserar på att använda för närvarande tillgängliga tekniker för att designa de kärnmoduler som behövs för att förbättra kapaciteten hos dessa system. % (denna mening säger inte så mycket, det är ganska allmänt ... vad gör du egentligen? dvs utformar ett kalmanfilter för att göra xyz)
Analysen och relaterad implementering av både lokaliserings- och kartläggningsfunktioner för autonoma gräsklippare presenteras.
Heterogena sensorer och deras olika konfigurationer undersöks och ett Adaptive Extended Kalman Filter föreslås för att smälta samman deras mätningar.
Denna teknik förbättrar poseuppskattningen av den autonoma gräsklipparen, som sedan utnyttjas av kartläggningsmodulen.
Det valda tillvägagångssättet för den senare, baserat på Bayesians slutledning, lyckas uppdatera kunskapen om kartan baserat på direkta interaktioner med omgivningen.

% skissa konsekvenserna av din lösning
De slutliga resultaten belyser vikten av exakt lokalisering som den verkliga flaskhalsen för utvecklingen av nya funktioner.
Den förbättrade positions-uppskattningen gör det möjligt att definiera en virtuell gräns. Definitionen inte tillräckligt korrekt för att korrekt kartlägga förekomsten av objekt i miljön
Exempel på avancerade funktioner från den föreslagna konfigurationen är implementeringen av deterministiska täckningsalgoritmer och interaktionen med gräsmattaobjekt.



\com{Autonoma gräsklippare har varit tillgängliga för konsumenterna i mer än $20$ år nu.
Under denna tidsperiod har dock deras konfiguration och prestanda sett små förbättringar.

Med framsteg inom inbyggda enhetsberäkningar och sensorprestationer kan implementeringen anses vara föråldrad.
De tillgängliga funktionerna är begränsade och fortfarande utsatta för fel.
Dessutom kräver den installerade externa infrastrukturen som de förlitar sig på, dvs begränsningskabeln, underhåll.

Denna avhandling analyserar implementeringen av en lokaliserings- och kartläggningsmodul.
Flera sensorer och deras olika konfigurationer analyseras för att förbättra poseuppskattningen av den autonoma gräsklipparen.
De bästa inställningarna antas sedan med kartmodulen för att uppdatera en karta med förståelse för robotens miljö.

Denna lösning är utgångspunkten för mer avancerade funktioner avancerade aspekter.
Tillgängligheten av en mer exakt lokalisering och kartdefinition möjliggör implementering av täckningsdefinitionsalgoritmer och interaktion med gräsmattan.
}
\subsection*{Nyckelord}
\noindent
Autonom mobil robot, sensorfusion, lokalisering, kartläggning

  \end{abstract}
\end{otherlanguage}


%\cleardoublepage
\clearpage
