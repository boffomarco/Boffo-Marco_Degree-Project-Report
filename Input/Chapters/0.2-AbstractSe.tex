
\begin{otherlanguage}{swedish}

\begin{abstract}
    \markboth{\abstractname}{}

\com{
    \todo[inline]{All theses at KTH are required to have an abstract in both English and Swedish.\\
If you are writing your thesis in English, you can leave this until the final version. If you are writing your thesis in Swedish then this should be done first – and you should revise as necessary along the way.\\
If you are writing your thesis in English, then this section can be a summary targeted at a more general reader. However, if you are writing your thesis in Swedish, then the reverse is true – your abstract should be for your target audience, while an English summary can be written targeted at a more general audience.\\
This means that the English abstract and Swedish sammnfattning
or Swedish abstract and English summary need not be literal translations of each other.\\

The abstract in the language used for the thesis should be the first abstract, while the Summary/Sammanfattning in the other language can follow.\\

Exchange students many want to include one or more abstracts in the language(s) used in their home institutions to avoid the neeed to write another thesis when returing to their home institution.
}
}

DRAFT

\noindent
Autonoma gräsklippare har varit tillgängliga för konsumenter i mer än $ 20 $ år nu.
Under denna tid har dock deras konfiguration och prestanda sett små förbättringar.

Med framstegen inom inbäddade enheter och sensorer är en sådan konfiguration nu föråldrad.
Funktionerna som tillhandahålls nu är begränsade och fortfarande utsatta för fel.
Dessutom kräver den installerade externa infrastrukturen som de förlitar sig på, dvs. begränsningskabeln, underhåll.

Denna avhandling analyserar implementeringen av en lokaliserings- och kartläggningsmodul som möjliggör implementering av fler funktioner.
Flera sensorer för att förbättra lokaliseringen av den autonoma gräsklipparen analyseras och den bästa konfigurationen presenteras.
Dessutom implementeras kartläggningsfunktioner över positioneringssystemet för att uppdatera förståelsen för robotens miljö.

Denna lösning kan användas för att förbättra täckningsfunktionen för den autonoma gräsklipparen.
Dessutom kan tillgängligheten av en mer exakt lokalisering vara utgångspunkten för mer avancerade funktioner, t.ex. interaktion med gräsmattan.

\subsection*{Nyckelord}
\noindent
Autonom mobil robot, sensorfusion, lokalisering, kartläggning

  \end{abstract}
\end{otherlanguage}


%\cleardoublepage
\clearpage
